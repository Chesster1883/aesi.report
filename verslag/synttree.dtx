%% \CheckSum{747}
% \iffalse meta-comment
%% synttree.dtx 
%% Package `synttree' for use with LaTeX 2e
%% Version 1.4.1
% 
% See the section "Author and License" below for copyright and licensing
% information 
% 
% \fi
%
%% \CharacterTable
%%  {Upper-case    \A\B\C\D\E\F\G\H\I\J\K\L\M\N\O\P\Q\R\S\T\U\V\W\X\Y\Z
%%   Lower-case    \a\b\c\d\e\f\g\h\i\j\k\l\m\n\o\p\q\r\s\t\u\v\w\x\y\z
%%   Digits        \0\1\2\3\4\5\6\7\8\9
%%   Exclamation   \!     Double quote  \"     Hash (number) \#
%%   Dollar        \$     Percent       \%     Ampersand     \&
%%   Acute accent  \'     Left paren    \(     Right paren   \)
%%   Asterisk      \*     Plus          \+     Comma         \,
%%   Minus         \-     Point         \.     Solidus       \/
%%   Colon         \:     Semicolon     \;     Less than     \<
%%   Equals        \=     Greater than  \>     Question mark \?
%%   Commercial at \@     Left bracket  \[     Backslash     \\
%%   Right bracket \]     Circumflex    \^     Underscore    \_
%%   Grave accent  \`     Left brace    \{     Vertical bar  \|
%%   Right brace   \}     Tilde         \~}
%%
%\iffalse    This is a METACOMMENT
%
% Version:  Date:	Changes:
% 0.9	    1998/07/19	First version as a .dtx distibution.
% 0.9a      ?		?
% 1.0	    ?		?
% 1.0a	    1999/10/02  Bugfix: bt combination changed to x.
% 1.1	    1999/10/13  More children, better code.
% 1.2	    2001	Totally new system
% 1.3	    2004/11/07	New system up and running.
% 1.3.1	    2004/11/07	Prepared for packaging, clarified licensing.
%      	    2004/12/30	Ignore leading and trailing spaces in labels.
%	    2005/04/13	Allow up to 10 children (or any number, in theory).
% 1.4	    2005/05/01	Allow any number of children, just warn.
%			Updated documentation.
% 1.4.1	    2006/07/20	Remove spurious spaces in output.
%			Simplify label typesetting.
%
%<*driver>
\documentclass[a4paper]{ltxdoc}
\usepackage[specials]{synttree}
\EnableCrossrefs         
 %\DisableCrossrefs     % Say \DisableCrossrefs if index is ready
\CodelineIndex
\RecordChanges          % Gather update information
 %\OnlyDescription      % comment out for implementation details
 \OldMakeindex         % use if your MakeIndex is pre-v2.9
\setlength\hfuzz{15pt}  % dont make so many
\hbadness=7000          % over and under full box warnings
\begin{document}
   \DocInput{synttree.dtx}
\end{document}
%</driver>
%
%\fi
%\MakeShortVerb{\"} 
% \title{The "synttree" package for typesetting syntactic
% trees.\footnote{Package version 1.4.1}}
% \author{Matijs van Zuijlen\footnote{e-mail: \texttt{mvz@xs4all}; web:
% \texttt{http://www.matijs.net/}}}
%
% \maketitle
%
% \begin{abstract}
%   The "synttree" package provides a simple way to typeset syntactic
%   trees as used in Chomsky's Generative Grammar.
% \end{abstract}
%
% {\parskip 0pt                ^^A We have to reset \parskip
%                              ^^A (bug in \LaTeX)
% \tableofcontents
% }
%  \section{Introduction}
%
%  The "synttree" package provides a macro for creating syntactic tree
%  structures such as those used in Noam Chomsky's Generative Grammar.
%  It is designed to create a tree that looks nice, without manual
%  tweaking, and with as little use of `special effects', such as
%  PostScript, as possible.
%
%  Since the application is very specific, there is no need for a very
%  complex set of options: "synttree" is designed to do one thing, and do
%  it well.
%
%  Using "synttree", you can create trees of any depth, and an unlimited
%  number of branches (up to the memory limit of your version of \TeX, of
%  course). 
%
%  \section{Usage}
%
%  \subsection{Drawing Options}
%
% The current implementation can use either em\TeX{} "\special"'s or
% \LaTeXe{}'s "\qbezier" macro to draw the lines in the tree,
% although the latter option is probably technically not very nice. Still,
% the behavior defaults to the \LaTeXe{} version.
%
%  In order to switch between the two modes, the following options are
%  supported:
%
%  \begin{description}
%
%  \item[specials] Use em\TeX{} special commands.
%  \item[nospecials] Don't use em\TeX{} special commands. This is the
%  default.
%
%  \end{description}
%
%  \subsection{Defined Macros}
%
%  \DescribeMacro{\synttree}
%  The main macro defined by this package is the "\synttree" macro. It
%  takes as arguments first an optional parameter indicating the
%  maximum depth of the tree, and second a parameter describing the
%  structure of the tree, delimited by square brackets ("[" and "]").
%
%  A tree consists of the parent node's label, followed by the child trees.
%  In principle, an unlimited number of children can be specified, but
%  to signal runaway trees, a warning will be issued when the number is
%  more than ten.
%  Each of the child trees is again surrounded by square brackets.
%  For example, the line
%\begin{verbatim}
%\synttree[A[B][C]]
%\end{verbatim}
%  creates the following tree:
% \begin{trivlist}
%  \item \synttree[A[B][C]]
% \end{trivlist}
%
%  Note that leading and trailing spaces in the label will be ignored, so
%  the following has the same result:
%\begin{verbatim}
%\synttree[ A [ B ][ C ]]
%\end{verbatim}
%^^A  Nodes and subtrees are surrounded by square
%^^A  brackets ("[" and "]"), as is the entire tree.
%
%  Before the label, a parameter may be added. This is specified
%  by appending a period (".") to the opening bracket, and appending a letter
%  that modifies the appearance of the tree. Append a "t" to
%  create a triangle instead of a line going from the label to the
%  level above it, and a "b" to specify that the node has to appear at
%  the bottom of the tree, on a line with the lowest leaves. To
%  use "b", a maximum tree depth has to be specified. To combine the
%  two features, use "x".
%
%  Using the optional parameters, the code
%\begin{verbatim}
%\synttree{4}
%[A
%  [B [.x Some text] ]
%  [D [E] [F[.t G]] [H] ]
%]
%\end{verbatim}%
%  creates this tree:
% \begin{trivlist}
%  \item
% \synttree{4}
% [A
%   [B [.x Some text] ]
%   [D [E] [F[.t G]] [H] ]
% ]
% \end{trivlist}
%
%  If brackets are included in the label, surround
%  them (or the entire label) by braces:
%
%\begin{verbatim}
%  \synttree[A[{B]D}][C]]
%\end{verbatim}
% \begin{trivlist}
%  \item \synttree[
%      A[{B]D}][C]]
% \end{trivlist}
%
%  The other defined macros affect the look and sizing of the trees.
%  The distance between levels can be adjusted by using
%  \DescribeMacro{\branchheight}
%  "\branchheight". It takes a length as its argument, and sets the
%  distance to that length. The default distance is half an inch. This
%  value is the distance between the baselines of the labels, so setting it
%  to zero will cause everything to be on one line.
%  The minimum horizontal separation between edges of two children is set
%  by
%  \DescribeMacro{\childsidesep}
%  "\childsidesep". The default is "1em".
%  The length
%  \DescribeMacro{\childattachsep}
%  "\childattachsep" indicates the minimum distance between the
%  attachment points of two children. Here, the default is "0.5in".
%  Finally, the balancing of the triangles can be adjusted by using
%  \DescribeMacro{\trianglebalance}
%  "\trianglebalance". It takes a number from 0 to 100, indicating the
%  percentage of the triangle on the right side of its attachment point to
%  the parent. The default is 50, resulting in isosceles triangles.
%
%  \section{Wish List and Bugs}
%
%  It is important to note that this package may not prevent you from doing
%  certain things `wrong'. For example, triangles on non-leaf nodes are
%  not forbidden, but may not give the desired result.
%
%  Sometimes, the label of a parent may be a phrase, and it may be
%  desirable to attach the children to a particular element in that phrase
%  (e.g., to elaborate on the structure of that particular element).  This
%  is currently not possible: The children are always attached to the
%  center of the label. Support for this may be added sooner or later.
%
%  \section{Related Packages}
%  A more complex package is the TreeTeX
%  system\cite{TreeTeX}. This system, however, produces nodes
%  consisting of a node symbol \emph{and} a label, whereas in
%  syntactic trees the label \emph{is} the node symbol. Additionally,
%  the method of specifying the tree structure itself makes the source
%  code hard to read.
%
%  The "xyling" package\cite{xyling} is a very powerful package, allowing
%  many things that "synttree" will not do. However, the nodes have to be
%  laid out meticulously in an array. Once that's done, however, the result
%  looks very good.
%
%  The ``\LaTeX\ for Linguists'' website\cite{LfL} lists other tree-drawing
%  packages, with comments on their usefulness.
%
%  \section{Author and License}
%
% Copyright (c) 1998, 1999, 2001, 2004--2006 Matijs van Zuijlen.
%
% The "synttree" package is free software. It has been released under the
% terms of the Latex Project Public License, version 1.3a. 
%
% This work has the LPPL maintenance status `maintained'. Its current
% maintainer is Matijs van Zuijlen. This work consists of the files
% "synttree.dtx" and "synttree.ins".
%
% \StopEventually{
%  \begin{thebibliography}{1}
%  \bibitem{LfL} Doug Arnold
%  \newblock {\it \LaTeX\ for Linguists} \\
%  \newblock {At \texttt{http://www.essex.ac.uk/linguistics/clmt/latex4ling/}}
%  \bibitem{TreeTeX} A. Br{\"u}ggemann-Klein and D. Wood.
%  \newblock {\it Drawing trees nicely with \TeX{}}
%  \newblock {File \texttt{tree\_doc.tex} in the "treetex" package}
%  \bibitem{BasicControl} Jonathan Fine.
%  \newblock {\it Some Basic Control Macros for \TeX{}}
%  \newblock {TUGboat, Volume 13 (1992), No. 1}
%  \bibitem{xyling} Ralf Vogel
%  \newblock {\it xyling --- \LaTeX\ macros for linguistic graphics} \\
%  \newblock {At \texttt{http://www.ling.uni-potsdam.de/~rvogel/xyling/}}
%  \end{thebibliography}
% }
%
% \section{Implementation}
%
% \subsection{Utility macros}
%
% Very useful for putting stuff after a "\fi" (as the name implies)
% \cite{BasicControl}.
%    \begin{macrocode}
\def\@AfterFi#1\fi{\fi#1}
\def\@AfterElseFi#1\else#2\fi{\fi#1}
%    \end{macrocode}
%
% \subsection{Drawing commands}
%
% These command are used to draw the lines between the nodes. There are two
% versions: one that uses the \LaTeXe{} "\qbezier" command, and one that is
% an adaptation of the unsupported "eepic" package, and uses specials.
% Currently, the package just uses the \LaTeXe{} version.
%
% First, the line drawing macros themselves. These simply draw a line
% between the two points given. Arguments are counters.
%    \begin{macrocode}
\def\MTr@latexdrawline(#1,#2)(#3,#4){%
  {%
    \count0=#1 \advance\count0 by #3 \divide\count0 2
    \count1=#2 \advance\count1 by #4 \divide\count1 2
    \qbezier(#1,#2)(\count0,\count1)(#3,#4)%
  }%
}
\def\MTr@etexdrawline(#1,#2)(#3,#4){%
  {%
    \count0=\@wholewidth \divide\count0 by 4736
    \special{pn \the\count0}%
    \count0= #1\advance \count0 2368 \divide \count0 4736
    \count1=-#2\advance \count1 -2368 \divide \count1 4736
    \special{pa \the\count0 \space \the\count1}%
    \count0= #3\advance \count0 2368 \divide \count0 4736
    \count1=-#4\advance \count1 -2368 \divide \count1 4736
    \special{pa \the\count0 \space \the\count1}%
    \special{fp}%
  }%
}
%    \end{macrocode}
%
% Options to select either version:
%    \begin{macrocode}
\DeclareOption{specials}{
  \let\MTr@drawline\MTr@etexdrawline%
}
\DeclareOption{nospecials}{
  \let\MTr@drawline\MTr@latexdrawline%
}
\ExecuteOptions{nospecials}%
\ProcessOptions%
%    \end{macrocode}
%
% \subsection{Definitions}
%
%  Some counters etc.\ are defined: The current level, the number of
%  children the current node has, the maximum level specified, also, the
%  current "branchmult", and whether the current node should be typeset
%  with a triangle. We need a separate counter to hold the current
%  "branchmult", because we calculate it before parsing the children, which
%  will each set "branchmult" globally to save it into the child registers.
%    \begin{macrocode}
\newcount\MTr@level
\newcount\MTr@numchildren
\newcount\MTr@maxlevel
\newcount\MTr@mybranchmult
\newif\ifMTr@mytriangle
%    \end{macrocode}
%
% A boolean to specify whether read tokens that are not "[" or "]" should
% be used as a label for a node, or just ignored.
%    \begin{macrocode}
\newif\ifMTr@uselabel
%    \end{macrocode}
%
% Two "savebox"es, one for the label, one to put the child in while it is
% being defined.
% ^^A There was: , and a token register, to put the label in before it is
% ^^A ready.
%    \begin{macrocode}
\newbox\MTr@labelbox
\newbox\MTr@treebox
%    \end{macrocode}
%
% \subsection{Storage for information on (sub)trees}
%
% Define storage space to store contents of and information about a tree's
% children. Distances are measured from the child's label's virtual
% attachment point to the line to its parent --- in reality, for aesthetic
% reasons, the line is drawn up to a point that lies slightly higher.
% Drawing a box around the whole child tree, we find the distance from the
% left side to the central point, $v$, the distance from the right side,
% $w$. Then, we have the height (equal to the height of the label), and the
% depth (the height of the box minus the height of the label).  There are
% also two external dimensions: the vertical distance from the parent
% label, $y$, and the horizontal distance from the parent tree's box, $x$.
% Usually, a child node is typeset one level below its parent.  However, we
% may set it lower, e.g., to set the child on the bottom level.
% "branchmult" indicates how many levels below the parent the child should
% be typeset.
%    \begin{macrocode}
\def\MTr@makechildcounter#1{%
    \expandafter\newcount\csname MTr@child#1\endcsname%
}
\def\MTr@makechildstoreage#1{%
    \expandafter\newsavebox\csname MTr@child#1box\endcsname%
    \MTr@makechildcounter{#1x}%
    \MTr@makechildcounter{#1y}%
    \MTr@makechildcounter{#1v}%
    \MTr@makechildcounter{#1w}%
    \MTr@makechildcounter{#1height}%
    \MTr@makechildcounter{#1depth}%
    \MTr@makechildcounter{#1branchmult}%
    \MTr@makechildcounter{#1picheight}%
    \MTr@makechildcounter{#1triangle}%
}
%    \end{macrocode}
% 
% \begin{macro}{\MTr@childparam}
% We also need an easy way to retrieve each child's parameters. This way,
% we don't have to use the "\csname" construct all the time.
%    \begin{macrocode}
\def\MTr@childparam#1#2{\csname MTr@child#1#2\endcsname}
%    \end{macrocode}
% \end{macro}
%
% Store info for current subtree: $v$, $w$ ($x$ and $y$ are external, thus
% not important), height, depth, "branchmult", triangle status.
%    \begin{macrocode}
\newcount\MTr@treev
\newcount\MTr@treew
\newcount\MTr@treeheight
\newcount\MTr@treedepth
\newcount\MTr@branchmult
\MTr@branchmult 1
\newif\ifMTr@triangle
%    \end{macrocode}
% Next, the depth, height and half the width of the label.
%    \begin{macrocode}
\newcount\MTr@labeldepth
\newcount\MTr@labelheight
\newcount\MTr@labelhalfwidth
%    \end{macrocode}
% When drawing, we need "morex", to store extra shift to the right when
% drawing the children, and "parenty", the vertical position of the point
% where the line from the parent to the child starts. Also, we need the
% width and height of the picture being drawn.
%    \begin{macrocode}
\newcount\MTr@morex
\newcount\MTr@parenty
\newcount\MTr@picwidth
\newcount\MTr@picheight
%    \end{macrocode}
% Finally, we need a temporary length, and four temporary counters.
%    \begin{macrocode}
\newlength{\MTr@templength}
\newcount\MTr@loopcnta
\newcount\MTr@tempcnta
\newcount\MTr@tempcntb
\newcount\MTr@tempcntc
%    \end{macrocode}
%
% \subsection{Adjustable and other parameters}
% \begin{macro}{\branchheight}
% The user can set up several parameters.
% First, "\branchheight" will set the distance
% between levels. Default value is half an inch.
% ^^A TODO: let default value depend on font size.
%    \begin{macrocode}
\newcount\MTr@branchheight%
\newcommand{\branchheight}[1]{%
  \setlength{\MTr@templength}{#1}%
  \MTr@branchheight\MTr@templength%
}
\branchheight{.5in}%
%    \end{macrocode}
% \end{macro}
%
% \begin{macro}{\trianglebalance}Next, "\trianglebalance" will set the
% balancing of the triangle. Default value is $50$.
%    \begin{macrocode}
\newcount\MTr@trianglemultright%
\newcount\MTr@trianglemultleft%
\newcommand{\trianglebalance}[1]{%
  \MTr@trianglemultleft100%
  \MTr@trianglemultright#1%
  \advance\MTr@trianglemultleft-#1%
}
\trianglebalance{50}%
%    \end{macrocode}
% \end{macro}
% Distance between the labels and the lines.
%    \begin{macrocode}
\newcount\MTr@lineoffset
\setlength{\MTr@templength}{2pt}%
\MTr@lineoffset\MTr@templength%
%    \end{macrocode}
% Minimum label height.
%    \begin{macrocode}
\newlength{\MTr@minheight}
\setlength{\MTr@minheight}{8pt}%
%    \end{macrocode}
% \begin{macro}{\childsidesep} Minimum separation between edges of two
% children.
%    \begin{macrocode}
\newcount\MTr@childsidesep
\newcommand{\childsidesep}[1]{%
  \setlength{\MTr@templength}{#1}%
  \MTr@childsidesep\MTr@templength%
  \ignorespaces%
}
\childsidesep{1em}
%    \end{macrocode}
% \end{macro}
% \begin{macro}{\childattachsep}How far apart are the attachment points of
% two children?
%    \begin{macrocode}
\newcount\MTr@childattachsep
\newcommand{\childattachsep}[1]{%
    \setlength{\MTr@templength}{#1}%
    \MTr@childattachsep\MTr@templength%
    \ignorespaces%
}
\childattachsep{0.5in}
%    \end{macrocode}
% \end{macro}
% 
% \subsection{Main macro}
% \begin{macro}{\synttree}
% "\synttree" is the main macro. 
% If the user has not provided a maximum depth, set it to 0. There
% will be no messages concerning depth, except when the bottomlevel
% modifier is used.
% Control is passed on to "\MTr@synttree"
%    \begin{macrocode}
\def\synttree{%
    \@ifnextchar[{\MTr@synttree{0}}{\MTr@synttree}%]
}
%    \end{macrocode}
% \end{macro}
%
% \begin{macro}{\MTr@synttree}
% This macro sets maximum depth, end sets picture coordinates to scaled
% points. Next, initial values of some variables are set, and the first
% real parsing macro, "\MTr@parserightbracket", is called.
%    \begin{macrocode}
\def\MTr@synttree#1{%
  \MTr@maxlevel#1%
  \unitlength 1sp%
  \MTr@level=0%
  \MTr@numchildren=0%
  \MTr@uselabelfalse%
  \MTr@parserightbracket%
}
%    \end{macrocode}
% \end{macro}
%
% \subsection{Parsing}
% \iffalse[[[\fi^^A Brackets needed for bracket balancing in text editor
%
% Parsing will only work if at least one "]" is present.
% \begin{macro}{\MTr@parserightbracket}
% First, scan until the first occurrance of "]". Argument "#1" will contain no
% "]", but may contain some "[", which are detected by
% "\MTr@parseleftbracket". After "#1" has been processed, lower the nesting
% level. If we're back at 0, we're done. Otherwise, continue looking for
% "]".
%    \begin{macrocode}
\def\MTr@parserightbracket#1]{%
  \MTr@parseleftbracket#1[:\END%
  \advance\MTr@level by -1%
  \MTr@dorightbracket%
  \ifnum\MTr@level=0%
    \unhbox\MTr@childibox{}%
  \else
    \@AfterFi{\MTr@parserightbracket}%
  \fi%
}
%    \end{macrocode}
% \end{macro}
% \begin{macro}{\Mtr@parseleftbracket}
% ^^A Scan after we already had one or more children: Just ignore anything
% ^^A that is not a "[" or "]".
% Scan until the first occurrance of "[". Argument "#1" will contain no "]"
% or "[". "#2" may contain more "]".
% Possibly, "#1" is a label.
% If "#2"=":", the "[" was placed by "\MTr@parserightbracket", and we're
% done.
%    \begin{macrocode}
\def\MTr@parseleftbracket#1[#2\END{%
  \ifMTr@uselabel%
    \MTr@bottomnodefalse%
    \MTr@mytrianglefalse%
    \MTr@parsedot#1.: \END%
  \fi%
  \ifx:#2%
  \else%
    \MTr@doleftbracket%
    \advance\MTr@level by 1%
    \@AfterFi{\MTr@parseleftbracket#2\END}%
  \fi%
}
%    \end{macrocode}
% \end{macro}
% \begin{macro}{\Mtr@parsedot}
% Parse the optional node argument.
%^^A TODO: Maybe this can be done simpler by using "\futurelet" in some way.
%    \begin{macrocode}
\def\MTr@parsedot#1.#2 #3\END{%
  \ifx:#2%
    \setbox\MTr@labelbox\hbox{\ignorespaces#1\unskip}%
  \else
    \ifx#2b\MTr@bottomnodetrue\else%
    \ifx#2x\MTr@bottomnodetrue\MTr@mytriangletrue\else%
    \ifx#2t\MTr@mytriangletrue\else%
    \typeout{synttree Warning: unknown dot option #1 in tree}%
    \fi\fi\fi%
    \MTr@parsedot#3\END
  \fi
}
%    \end{macrocode}
% \end{macro}
%
% \subsection{Action}
%
% The next macros implement the actions to be undertaken when encountering
% one of the square brackets.
%
% \begin{macro}{\MTr@doleftbracket}
% Upon encountering a "[", we go one level deeper: Begin a group, and reset
% parameters.
%    \begin{macrocode}
\def\MTr@doleftbracket{%
  \bgroup%
  \MTr@numchildren=0%
  \MTr@uselabeltrue%
}
%    \end{macrocode}
% \end{macro}
%
% \begin{macro}{\MTr@dorightbracket}
% Upon encountering a "]", we typeset the current group's tree, and end the
% group. Until we encounter some brackets, we should not add tokens to any
% label. Save the just-typeset tree as the newest child in the parent group
% --- now the current group.
% ^^A TODO: Put this back into parserightbracket.
%    \begin{macrocode}
\def\MTr@dorightbracket{%
  \MTr@maketreebox%
  \egroup%
  \MTr@uselabelfalse%
  \MTr@savecurrentchildbox%
}
%    \end{macrocode}
% \end{macro}
% \begin{macro}{\MTr@savecurrentchildbox}
% Save the current picture, with all its data, in the box and registers for
% the next empty child.
%    \begin{macrocode}
\def\MTr@savecurrentchildbox{%
  \advance\MTr@numchildren by 1
  \ifnum\MTr@numchildren<1%
    \typeout{synttree internal warning: There is no child box to save.}%
  \else
    \ifnum\MTr@numchildren>10%
      \typeout{synttree warning: More than 10 child boxes.
      Can this be true?}%
    \fi
    \MTr@savechildbox{\romannumeral\MTr@numchildren}%
  \fi
}
%    \end{macrocode}
% \end{macro}
%
% \begin{macro}{\MTr@savechildbox}
% Save current tree in a specific child box. Basically, we just copy from
% the tree registers to the child registers, except $x$ is set to $v$
% initially. Used only in "\MTr@savecurrentchildbox". 
%    \begin{macrocode}
\def\MTr@savechildbox#1{%
  \expandafter
  \ifx\csname MTr@child#1box\endcsname\relax%
    \MTr@makechildstoreage{#1}%
  \fi%
  \setbox%
  \csname MTr@child#1box\endcsname%
  \hbox{\unhbox\MTr@treebox}%
  \csname MTr@child#1v\endcsname\MTr@treev%
  \csname MTr@child#1w\endcsname\MTr@treew%
  \csname MTr@child#1x\endcsname\MTr@treev%
  \csname MTr@child#1height\endcsname\MTr@treeheight%
  \csname MTr@child#1depth\endcsname\MTr@treedepth%
  \csname MTr@child#1branchmult\endcsname\MTr@branchmult%
  \ifMTr@triangle%
    \csname MTr@child#1triangle\endcsname 1%
  \else%
    \csname MTr@child#1triangle\endcsname 0%
  \fi
}
%    \end{macrocode}
% \end{macro}
%
% \subsection{Bottom Nodes}
% 
% For bottom nodes, we have to adapt the vertical position so that
% they become, indeed, bottom nodes. This is done by the macros
% "\MTr@bottomnodetrue" and "\MTr@bottomnodefalse".
%
% \begin{macro}{\MTr@bottomnodetrue}
% The node is a bottom node. Calculate the difference between this
% level and the bottom level as passed to "\synttree", and use this to
% determine how many levels the node has to be advanced vertically to
% get it at the correct position. In effect, the distance between two
% levels is multiplied by the ``branch multiplication.'' For
% non-bottom nodes this is set to 1.
% ^^A TODO: always check levels if a maxlevel has been specified.
%    \begin{macrocode}
\def\MTr@bottomnodetrue{%
  \MTr@branchmult\MTr@maxlevel%
  \advance\MTr@branchmult-\MTr@level%
  \advance\MTr@branchmult 1%
  \ifnum\MTr@branchmult<1%
    \typeout{synttree Warning: Tree has more levels than indicated.}%
    \typeout{>> Indicated: \the\MTr@maxlevel.}%
    \typeout{>> Level now: \the\MTr@level.}%
    \MTr@branchmult1%
  \fi%
  \MTr@mybranchmult\MTr@branchmult%
}
%    \end{macrocode}
% \end{macro}
%
% \begin{macro}{\MTr@bottomnodefalse}
% It's not a bottom node: Just set the branch multiplication to 1.
%    \begin{macrocode}
\def\MTr@bottomnodefalse{%
  \MTr@mybranchmult1%
}
%    \end{macrocode}
% \end{macro}
% 
% \subsection{Utility macros for the output routines}
%
% \begin{macro}{\MTr@setverticalchilddimens}
% Set vertical dimensions for a given child: $y$ and "picheight".
% ^^A TODO: Maybe we can set these while saving the child.
%    \begin{macrocode}
\def\MTr@setverticalchilddimens#1{%
    \MTr@tempcnta-\MTr@branchheight%
    \multiply\MTr@tempcnta\MTr@childparam{#1}{branchmult}%
    \MTr@tempcntb-\MTr@tempcnta%
    \advance\MTr@tempcntb\csname MTr@child#1depth\endcsname%
    \advance\MTr@tempcnta-\MTr@labelheight%
    \advance\MTr@tempcnta\csname MTr@child#1height\endcsname%
    \csname MTr@child#1y\endcsname\MTr@tempcnta%
    \csname MTr@child#1picheight\endcsname\MTr@tempcntb%
}
%    \end{macrocode}
% \end{macro}
%
% \begin{macro}{\MTr@adjustdistance}
% Adjust the distance between two children. For the pair of
% neighboring children, "tempcnta" is used temporarily to store what
% amounts to the distance between the central points of the two children
% (the left child's $w$, plus the right child's $v$).
% If it is below the value "childattachsep", adjust values so that is equal
% to it. "tempcnta" stores the resulting value.
% The right child's $x$ is now set to put it in the correct position
% w.r.t.\  the left child.
%    \begin{macrocode}
\def\MTr@adjustdistance#1#2{%
  \MTr@tempcnta\MTr@childparam{#1}{w}%
  \advance\MTr@tempcnta\MTr@childsidesep%
  \advance\MTr@tempcnta\csname MTr@child#2v\endcsname%
  \ifnum\MTr@tempcnta<\MTr@childattachsep%
    \MTr@tempcnta\MTr@childattachsep%
  \fi%
  \csname MTr@child#2x\endcsname\MTr@childparam{#1}{x}%
  \advance\csname MTr@child#2x\endcsname\MTr@tempcnta%
}
%    \end{macrocode}
% \end{macro}
% \begin{macro}{\MTr@setparentdimens}
% Set some parameters for the parent node. Parameters are the leftmost and
% rightmost child label.
%    \begin{macrocode}
\def\MTr@setparentdimens#1#2{%
%    \end{macrocode}
% Calculate the subtree's $v$ and $w$:
%    \begin{macrocode}
    \MTr@tempcnta\MTr@childparam{#2}{x}%
    \advance \MTr@tempcnta -\MTr@childparam{#1}{x}%
    \divide\MTr@tempcnta 2%
    \MTr@treev\MTr@tempcnta%
    \MTr@treew\MTr@treev%
    \advance \MTr@treev \csname MTr@child#1x\endcsname%
    \advance \MTr@treew \csname MTr@child#2w\endcsname%
%    \end{macrocode}
% Calculate "morex": The distance the (first) subtree has to be shifted to
% the right to accomodate a large parent label size.
%    \begin{macrocode}
    \MTr@morex\MTr@labelhalfwidth%
    \advance\MTr@morex-\MTr@treev%
    \ifnum\MTr@morex<0\MTr@morex0\fi%
%    \end{macrocode}
% Large label sizes are also incorporated into $w$ and $x$.
%    \begin{macrocode}
    \ifnum\MTr@treew<\MTr@labelhalfwidth
        \MTr@treew\MTr@labelhalfwidth
    \fi%
    \ifnum\MTr@treev<\MTr@labelhalfwidth
        \MTr@treev\MTr@labelhalfwidth
    \fi%
%    \end{macrocode}
% Picture width.
%    \begin{macrocode}
    \MTr@picwidth\MTr@treev%
    \advance\MTr@picwidth\MTr@treew%
}
%    \end{macrocode}
% \end{macro}
% \begin{macro}{\MTr@setpictureparameters}
% Set some parameters for the picture.
%    \begin{macrocode}
\def\MTr@setpictureparameters{%
    \global\MTr@treedepth\MTr@picheight%
    \advance\MTr@picheight\MTr@labelheight%
    \global\MTr@treeheight\MTr@labelheight%
    \MTr@parenty-\MTr@labelheight%
    \advance\MTr@parenty-\MTr@labeldepth%
    \advance\MTr@parenty-\MTr@lineoffset%
    \global\MTr@treev\MTr@treev%
    \global\MTr@treew\MTr@treew%
}
%    \end{macrocode}
% \end{macro}
% \begin{macro}{\MTr@drawlabel}
%    \begin{macrocode}
\def\MTr@drawlabel{%
  \put(\MTr@treev,0){%
    \makebox(0,0)[t]{%
      \rule{0pt}{\MTr@minheight}%
      \usebox{\MTr@labelbox}}}%
}
%    \end{macrocode}
% \end{macro}
% \begin{macro}{\MTr@drawchild}
% Draws the saved child node's picture.
%    \begin{macrocode}
\def\MTr@drawchild#1{%
  \MTr@tempcnta\MTr@childparam{#1}{x}
  \advance\MTr@tempcnta-\MTr@childparam{#1}{v}
  \put(\MTr@tempcnta,\MTr@childparam{#1}{y}){%
    \makebox(0,0)[tl]{%
      \usebox{\csname MTr@child#1box\endcsname}}}%
}
%    \end{macrocode}
% \end{macro}
% \begin{macro}{\MTr@drawchildline}
% Draws the line or triangle from the parent to the given child.
%    \begin{macrocode}
\def\MTr@drawchildline#1{
%    \end{macrocode}
% Use child's $y$, but advance it to make the line stop just above the
% label.
%    \begin{macrocode}
  \MTr@tempcnta\MTr@childparam{#1}{y}
  \advance\MTr@tempcnta\MTr@lineoffset%
  \expandafter
  \ifnum\csname MTr@child#1triangle\endcsname=1%
    \MTr@tempcntb\MTr@childparam{#1}{x}%
    \MTr@tempcntc\MTr@tempcntb%
    \advance\MTr@tempcntb \MTr@childparam{#1}{w}%
    \advance\MTr@tempcntc -\MTr@childparam{#1}{v}%
    \put(0,0){\MTr@drawline%
        (\MTr@treev,\MTr@parenty)%
        (\MTr@tempcntc,\MTr@tempcnta)}%
    \put(0,0){\MTr@drawline%
        (\MTr@treev,\MTr@parenty)%
        (\MTr@tempcntb,\MTr@tempcnta)}%
    \put(0,0){\MTr@drawline%
        (\MTr@tempcntc,\MTr@tempcnta)%
        (\MTr@tempcntb,\MTr@tempcnta)}%
  \else%
    \put(0,0){\MTr@drawline%
      (\MTr@treev,\MTr@parenty)%
      (\MTr@childparam{#1}{x},\MTr@tempcnta)}%
  \fi%
}
%    \end{macrocode}
% \end{macro}
%
% \subsection{Output routines}
%
% \begin{macro}{\MTr@maketreebox}
% This is the main macro that starts the output.
% ^^A TODO: Maybe move \global\setbox\MTr@treebox out of the \ifnum s
%    \begin{macrocode}
\def\MTr@maketreebox{%
  \MTr@labelheight\ht\MTr@labelbox%
  \ifnum\MTr@labelheight<\MTr@minheight\MTr@labelheight\MTr@minheight\fi%%
  \MTr@labeldepth\dp\MTr@labelbox%
  \MTr@labelhalfwidth\wd\MTr@labelbox%
  \divide\MTr@labelhalfwidth 2%
  \ifnum\MTr@numchildren=0%
    \global\setbox\MTr@treebox\hbox{\MTr@outputlabel}%
  \fi%
  \ifnum\MTr@numchildren>0%
    \global\setbox\MTr@treebox\hbox{\MTr@outputchildren{\the\MTr@numchildren}}%
  \fi%
  \global\MTr@branchmult\MTr@mybranchmult%
  \ifMTr@mytriangle%
    \global\MTr@triangletrue%
  \else%
    \global\MTr@trianglefalse%
  \fi%
}
%    \end{macrocode}
% \end{macro}
% \begin{macro}{\MTr@outputlabel}
% Output a tree that is just a label, with no children.
%    \begin{macrocode}
\def\MTr@outputlabel{%
%    \end{macrocode}
% First, set parameters: Height and depth of the subtree are equal to
% height and depth of the label. The $x$ and $w$ of the subtree each equal
% half the width of the label. Optionally, $x$ may be zero and $w$ may
% equal to the entire width of the label. The width and height of the
% picture to be drawn equal the width of the label and the height of the
% tree.
%    \begin{macrocode}
    \global\MTr@treeheight\MTr@labelheight%
    \global\MTr@treedepth\MTr@labeldepth%
    \ifMTr@mytriangle%
      \MTr@treew\MTr@labelhalfwidth%
      \MTr@treev\MTr@labelhalfwidth%
      \multiply\MTr@treew \MTr@trianglemultright%
      \multiply\MTr@treev \MTr@trianglemultleft%
      \divide\MTr@treew 50%
      \divide\MTr@treev 50%
      \global\MTr@treew\MTr@treew%
      \global\MTr@treev\MTr@treev%
    \else%
      \global\MTr@treew\MTr@labelhalfwidth%
      \global\MTr@treev\MTr@labelhalfwidth%
    \fi%
    \MTr@picwidth\wd\MTr@labelbox%
    \MTr@picheight\MTr@treeheight%
%    \end{macrocode}
% Second, draw the picture. The coordinates for the picture are nearly
% the same throughout. In any event, the label is centered in the
% picture, its baseline aligned with the bottom of the picture.
%    \begin{macrocode}
    \advance\MTr@picheight\MTr@treedepth%
    \begin{picture}%
        (\MTr@picwidth,\MTr@picheight)%
        (-\MTr@treev,-\MTr@picheight)%
      %\put(-\MTr@treev,-\MTr@picheight){\framebox(\MTr@picwidth,\MTr@picheight){}}%
      \put(-\MTr@treev,0){%
        \makebox(0,0)[tl]{%
          \rule{0pt}{\MTr@minheight}%
          \usebox{\MTr@labelbox}}}%
    \end{picture}%
}
%    \end{macrocode}
% \end{macro}
%
% \begin{macro}{\MTr@outputchildren}
% Output a tree or subtree with at least one child tree.
%    \begin{macrocode}
\def\MTr@outputchildren#1{%
%    \end{macrocode}
% Calculate desired distance between the children.
% The cumulative value is put into "treev".
%    \begin{macrocode}
    \ifnum#1>1
      \MTr@loopcnta 1
      \loop
	\edef\MTr@temp{\romannumeral\MTr@loopcnta}%
	\ifnum\MTr@loopcnta<#1
	  \advance \MTr@loopcnta by 1
	  \MTr@adjustdistance{\MTr@temp}{\romannumeral\MTr@loopcnta}%
      \repeat
    \fi
%    \end{macrocode}
% Calculate the subtree's $v$ and $w$, based on the children, and the
% label's width.
%    \begin{macrocode}
    \MTr@setparentdimens{i}{\romannumeral#1}%
%    \end{macrocode}
% After the call to "setparentdimens", all children's $x$
% values must be advanced by "morex". In the same loop, also
% set vertical position and picture height for each child.
% The height of the picture is the height of the largest child.
%    \begin{macrocode}
    \MTr@picheight 0%
    \MTr@loopcnta #1
    \loop
      \edef\MTr@temp{\romannumeral\MTr@loopcnta}%
      \advance\MTr@childparam{\MTr@temp}{x}\MTr@morex%
      \MTr@setverticalchilddimens{\romannumeral\MTr@loopcnta}%
      \ifnum\MTr@childparam{\MTr@temp}{picheight}>\MTr@picheight%
	\MTr@picheight\MTr@childparam{\MTr@temp}{picheight}%
      \fi%
    \advance \MTr@loopcnta by -1 \ifnum\MTr@loopcnta>0 \repeat
%    \end{macrocode}
% Set various other parameters.
%    \begin{macrocode}
  \MTr@setpictureparameters%
%    \end{macrocode}
% Start the picture and draw the label. Next, draw each child, and the line
% to that child. Then, end the picture.
%    \begin{macrocode}
  \begin{picture}(\MTr@picwidth,\MTr@picheight)(0,-\MTr@picheight)%
    %\put(0,-\MTr@picheight){\framebox(\MTr@picwidth,\MTr@picheight){}}%
    \MTr@drawlabel%
    \MTr@loopcnta #1
    \loop
      \expandafter\MTr@drawchild{\romannumeral\MTr@loopcnta}%
      \expandafter\MTr@drawchildline{\romannumeral\MTr@loopcnta}
    \advance \MTr@loopcnta by -1 \ifnum\MTr@loopcnta>0 \repeat
  \end{picture}%
}
%    \end{macrocode}
% \end{macro}
%
% \Finale
% \PrintIndex \PrintChanges
\endinput
